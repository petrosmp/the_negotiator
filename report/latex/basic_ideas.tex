\paragraph*{}
The main idea behind TheNegotiator is that we have no explicit strategy to propose, evaluate and accept offers, but we rather employ a set of other, peer-designed agents and pretend that we are one of them in each round. % TODO: etsi opws to lew fainetai san na eimaste diaforetikoi se kathe round. mporoume na eimaste kai idioi omws. we need that to come across
This shifts the problem that we have to solve from the bilateral negotiation one to the algorithm selection one: \quotes{Which algorithm out of a set of possible candidates is likely to perform best for a well-defined problem} \cite{original_alg_selection_paper_1975}. A great source of inspiration and a starting point for further research has been \cite{meta_agent_paper}, which also offers more background and theoretical insight into the algorithm selection problem in our setting.

\paragraph*{}
A key piece of insight here is \emph{when exactly are we asked to choose a strategy}: by competition rules we are allowed to change our strategy between negotiation sessions, but not during one.
In other words, we are free to pick the best strategy for every negotiation session right before it begins, as long as we stick with it during the whole session.

\paragraph*{}
Much like \cite{meta_agent_paper}, we approach the problem from a machine learning (ML) standpoint, using ML techniques to perform two key actions:
\begin{itemize}
    \item estimate the performance of the available strategies on a never-before-seen domain
    \item maintain and adjust that estimate throughout the competition, taking into consideration the results we get when using each strategy
\end{itemize}
